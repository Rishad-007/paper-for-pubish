\begin{abstract}
This study breaks new ground by blending macroeconomic trends with microeconomic firm-level data to uncover how broad fiscal policies (i.e., Value-Added Tax, VAT) trickle down to affect individual businesses' survival and operations—something rarely explored in such an integrated way before, marking our key contribution to more holistic fiscal policy analysis. Our approach deploys three powerful machine learning tools: Long Short-Term Memory (LSTM) networks to detect long-term patterns in economic time series like growth and inflation; Double Machine Learning (DoubleML) to eliminate biases from complex confounding factors and provide clean average impact estimates; and Causal Forests to identify effect heterogeneity across firm sizes, sectors, and conditions without needing predefined groups. This hybrid model enhances performance by leveraging each method’s strengths—LSTM captures economic cycle dynamics, DoubleML provides rigorous causal identification, and Causal Forests expose hidden variations, such as how larger firms in dense industries recover better while smaller ones suffer more during tight monetary conditions.

Prior research, such as  \citep{agrawal2024effects} and  \citep{singh2019impact}, mainly focused on aggregate effects of VAT adoption on firm revenues and production efficiency but lacked integrated causal techniques to uncover nuanced firm-level survival patterns and heterogeneity in responses. Studies like  \citep{bolarindwa2023vat} and analyses of SMEs in developing economies \citep{ndlovu2024vat} demonstrated correlations between VAT policy and reduced profitability or growth constraints yet did not exploit advances in causal machine learning to control for confounders or capture dynamic effects over economic cycles.

Our combined machine learning methodology improves upon these by delivering sharper predictive accuracy (RMSE = 0.0287, $R^{2}=0.895$) and clearer insights into how a 5\% VAT hike cuts firm survival by approximately 3.87 percentage points (semi-elasticity of -0.77 percentage points per 1\% VAT). This translates to roughly 22,800 adversely affected firms, particularly during downturns when policy sensitivity is heightened, while bold fiscal stimulus through tax cuts generates stronger rebounds. The findings underscore the power of tailored, data-driven fiscal approaches that protect vulnerable firms and promote economic resilience under challenging conditions.
\end{abstract}





\keywords{Value-Added Tax, Machine Learning, Causal Inference, LSTM, Causal Forest, Econometric Models, Economic Forecasting, Firm Survival, Policy Impact}



