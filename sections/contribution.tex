\section{Contributions}\label{sec:contributions}
This thesis makes four principal contributions to the domains of applied macroeconomic policy analysis, causal inference, and machine learning integration:



\begin{enumerate}[label=\textbf{C\arabic*.}]
  \item \textbf{Hybrid Analytical Framework}: Develops an end-to-end framework that integrates time-series forecasting (\LSTM) with heterogeneous treatment effect estimation (\CF) and econometric validation layers to assess fiscal policy impacts, specifically \VAT{} adjustments.
  \item \textbf{Methodological Synthesis}: Formalizes a modular pipeline combining feature engineering, sequential dependency modeling, counterfactual construction, and post-estimation interpretability (feature importance, partial dependence, subgroup analysis).
  \item \textbf{Empirical Insights}: Provides evidence on the macroeconomic and sectoral responses to \VAT{} policy changes, highlighting nonlinear dynamics, regime sensitivity, and distributional heterogeneity.
  \item \textbf{Policy Evaluation Toolkit}: Delivers reproducible components (data schema, model specifications, diagnostic procedures) that can be adapted for broader fiscal and regulatory policy assessment.
\end{enumerate}

Collectively, these contributions advance data-driven economic policy evaluation by unifying predictive accuracy, causal interpretability, and operational usability.
