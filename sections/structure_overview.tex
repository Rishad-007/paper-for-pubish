\section{Structure of the Thesis}\label{sec:structure}
The remainder of this thesis is organized as follows:
\begin{itemize}
  \item \textbf{Section~\ref{sec:literature}} reviews related literature on causal inference, machine learning applications in economics, and counterfactual policy analysis.
  \item \textbf{Section~\ref{sec:data}} describes the datasets, data sources, preprocessing workflow, and construction of analytical variables.
  \item \textbf{Section~\ref{sec:methodology}} presents the hybrid methodological framework, including model architectures, integration logic, estimation routines, and validation design.
  \item \textbf{Section~\ref{sec:validation}} outlines validation strategies, including cross-validation, performance metrics, robustness checks, and sensitivity analyses.
  \item \textbf{Section~\ref{sec:results}} reports empirical results with tables, figures, and comparative benchmarks across model classes.
  \item \textbf{Section~\ref{sec:discussion}} provides a detailed discussion of findings, policy implications, and comparative strengths of the hybrid framework.
  \item \textbf{Section~\ref{sec:conclusion}} concludes with a summary of contributions, limitations, and avenues for future research.
  \item \textbf{Appendices} supply supplementary material including extended tables, additional robustness diagnostics, and implementation notes.
\end{itemize}

